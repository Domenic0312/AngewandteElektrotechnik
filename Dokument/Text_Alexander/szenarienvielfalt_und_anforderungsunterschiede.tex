\section{Szenarienvielfalt und Anforderungsunterschiede}

In dem Modul `Projekt angewandte Elektrotechnik' bestand die Aufgabe darin ein Rescue Robot zu konzipieren.
Dabei handelt es sich um einen Roboter, der in Gefahrensituationen interagieren kann. 
Die Vorteile von einem autonomen oder auch ferngesteuerten Roboter sind groß. So kann dieser sich in Situationen begeben, die für einen Menschen oder einen Rettungshelfer zu gefährlich sind. Dabei sind die Aufgabenanforderungen und die mögliche Szenarien-Vielfalt quasi unendlich. 

\subsection {Szenarienvielfalt}
Mögliche Szenarien sind zum Beispiel Umweltkatastrophen. 
Nach einem Tsunami oder schweren Regenfällen sind Überschwemmungen die Folge.
Dabei können Erdmassen mitgetragen werden und neu entstandenen Flüsse in urbanen Gebieten entstehen. 
Wenn die Luftrettung oder eine generelle Rettungsmaßnahme von Rettungshelfern scheitert durch zu gefährliche Bedingungen, könnte ein Roboter helfen. 
Dieser kann an Orte gelangen die unzugänglich sind und den Kontakt zu Personen aufstellen bzw. einen Rettungsversuch einleiten. 
Dabei wäre er so konzipiert das dieser schwimmen kann und sich durch unklares fließendes Gewässer navigiert. 
Als contraire Szenario ist aber auch ein Waldbrand möglich oder auch ein Industriebrand.
Durch die vorherrschende Hitze und hohe Temperatur sind ganz andere Anforderungen an den Rescue Roboter gestellt. 
So müsste dieser an Land und über Hindernisse fahren können und hitzebeständig sein. 
Es sind aber auch noch andere Szenarien möglich, die für Rettungskräfte zu gefährlich sind. 
Dazu zählen zum Beispiel Chemie oder Atomunfälle. Wenn die Strahlung zu hoch ist, hat dies langanhaltende Folgen für Menschen.
Ein Roboter könnte in die Gefahrenzone hineinfahren. 
Über Kameras und Messinstrumente können wichtige Daten weitergeleitet werden. 
Hier müsste der Roboter speziell beschichtet sein, um der Strahlung standzuhalten. 

\subsection {Anforderungsunterschiede}
Es zeigt sich, dass es viele mögliche Szenarien für einen Rescue Roboter gibt. 
Dabei sind die Anforderungen höchst unterschiedlich und situationsbedingt. 
Jedes Einsatzgebiet erfordert andere Maßnahmen und technische Umsetzungen. 
Wünschenswert wäre ein Roboter, der alles kann. Da dies aktuell und zukünftig technisch nicht realisierbar ist, ist es sinnvoll sich auf ein Szenario zu fokussieren.
Durch die Spezialisierung wird der Roboter bestmöglich ausgestattet, um die Aufgaben zu erfüllen.






